\subsubsection{The Interplay of Rate and Latency Encoding}

To further enhance the bio-inspired character of our model and to leverage the properties of both rate and latency encoding, we propose a novel encoding scheme that combines both principles.

Rate encoding represents the intensity of a stimulus by the frequency of the generated spike train. A high frequency corresponds to a high-intensity stimulus and vice versa. Latency encoding, on the other hand, uses the time of the first spike after stimulus onset to represent stimulus intensity, with a shorter latency indicating a stronger stimulus \cite{thorpe2001spike}.

In our combined encoding scheme, we utilize latency encoding to represent the rapid changes in the stimulus intensity and rate encoding to carry information about the sustained level of stimulus intensity. This scheme allows us to capture both the temporal dynamics and the static properties of the input signal.

\subsubsection{Proposed SNN Architecture for Retina Modelling}

Our proposed SNN architecture models the structure of the retina, which consists of three layers of neurons: photoreceptors (cones and rods), bipolar cells, and ganglion cells.

\textbf{Photoreceptor layer}: The input to our network is converted into spike trains by photoreceptor neurons. The photoreceptor cells are modeled as Poisson neurons, whose firing rate is determined by the intensity of the input signal (rate encoding). In addition, the time of the first spike (latency encoding) carries information about the onset of the stimulus. Each photoreceptor cell thus produces a spike train that represents both the level and the changes in stimulus intensity.

\textbf{Bipolar cell layer}: The spike trains generated by the photoreceptor cells are fed into the bipolar cell layer. The synapses between the photoreceptor and bipolar cells are governed by the alpha function, which models the post-synaptic potential (PSP) in response to incoming spikes. The parameters of the alpha function ($\tau_m$ and $\tau_s$) can be adjusted to alter the temporal dynamics of the PSP, enabling us to capture the characteristics of the specific synaptic transmission.

\textbf{Ganglion cell layer}: The final layer of our network consists of ganglion cells. These neurons integrate the signals received from the bipolar cells and convert them into output spike trains. The output of our network can be taken as the firing rates of the ganglion cells, which carry information about the original input signal.

By structuring our SNN in this way, we can capture the layered architecture of the retina and exploit the properties of both rate and latency encoding to represent the input signal in a biologically plausible manner \cite{gollisch2008rapid}.