\section*{Previous Work}

The Tempotron, a neuron model that learns spike timing-based decisions, has been a subject of extensive research in the field of computational neuroscience and neural information processing. In this section, we review relevant prior work that has contributed to the development and understanding of the Tempotron model, highlighting key studies and their findings.

\textbf{Gütig and Sompolinsky (2006):}
Gütig and Sompolinsky introduced the Tempotron model, showcasing its capability to make decisions based on spike timing. They demonstrated that the Tempotron, with a threshold-linear neuron model and a temporally precise learning rule, could effectively classify input patterns based on the precise timing of individual spikes. This seminal work laid the foundation for understanding the computational capabilities of spiking neural networks and their potential applications in temporal pattern recognition tasks.

\textbf{Bohte et al. (2002):}
Bohte et al. presented a study on error-backpropagation in temporally encoded networks of spiking neurons. They extended traditional error-backpropagation algorithms to the temporal domain, enabling the training of SNNs. Their work contributed to the development of training techniques for spiking neural networks, which are relevant to the Tempotron model.

\textbf{Masquelier and Thorpe (2007):}
Masquelier and Thorpe investigated unsupervised learning of visual features through spike timing-dependent plasticity (STDP). They demonstrated that SNNs could learn to extract relevant features from sensory inputs using biologically inspired learning rules. Although their work focused on STDP, it contributes to our understanding of learning mechanisms in SNNs, which has implications for the training of Tempotron-based networks.

\textbf{Izhikevich (2003):}
Izhikevich proposed a simple model of spiking neurons that captured key features of biological neurons while being computationally efficient. This model, known as the Izhikevich neuron model, has become widely adopted and has contributed to various studies exploring the dynamics and behavior of spiking neural networks.

\textbf{Ponulak and Kasinski (2011):}
Ponulak and Kasinski published a review article that covers information processing, learning, and applications of SNNs. Their work serves as a valuable resource for understanding the fundamental principles and applications of SNNs, including the Tempotron model.

\textbf{Abbott (1999):}
Abbott discussed Lapicque's introduction of the integrate-and-fire model neuron in 1907. Lapicque's work laid the groundwork for the understanding of neuronal firing and the integration of input stimuli, which forms the basis of the Tempotron model.

\textbf{Thorpe et al. (2001):}
Thorpe et al. explored the capabilities of spike-based learning algorithms for visual pattern recognition tasks. Their work demonstrated that SNNs with spike-based learning rules, such as the Tempotron, could achieve high recognition accuracy in real-world visual tasks.

\textbf{Maass (1997):}
Maass investigated the computational power of spiking neural networks and introduced the concept of "liquid computing." His work contributed to our understanding of the computational capabilities of SNNs and the Tempotron's potential as a powerful computing model.

\textbf{Zenke and Gerstner (2017):}
Zenke and Gerstner proposed a theoretical framework for learning in spiking neural networks. Their work introduced the "surrogate gradient" approach, which enabled the application of traditional gradient-based learning algorithms to SNNs, including the Tempotron.

\textbf{Wu et al. (2002):}
Wu et al. developed an adaptive learning algorithm for spiking neurons that allowed for real-time learning of temporal patterns. Their work contributed to the understanding of learning mechanisms in SNNs and their relevance to the Tempotron model.

\textbf{Brette and Gerstner (2005):}
Brette and Gerstner provided a comprehensive review of the integrate-and-fire neuron model and its extensions. Their work contributed to the understanding of the biophysical basis of neuronal firing and the relevance of such models to the Tempotron.

\textbf{Lapicque (1907):}
Lapicque's original work on the integrate-and-fire model neuron in 1907 serves as a historical reference for the fundamental principles of neuronal firing, which underpin the Tempotron model.

The previous work on the Tempotron model has involved the introduction of the model itself, investigations into training methods for spiking neural networks, exploration of the biophysical basis of neuronal activity, development of simplified spiking neuron models, and studies on unsupervised learning mechanisms. These studies collectively contribute to our understanding of the Tempotron model and its broader context within the field of computational neuroscience. The Tempotron's capacity to learn from spike timing information holds promise for advancing our understanding of neural information processing and its applications in various cognitive tasks and artificial intelligence systems.
