\subsection{Population Encoding}

We shall now delve into various strategies to adopt the population encoding method. Our focus is to utilize this approach in a manner that leverages the salient features of the data, harnesses the unique characteristics of the neuron model, and employs the diverse encoding methodologies we have previously explored in \ref{ch:computational-models}.

As we discussed in section \ref{ch:computational-models}, we can integrate different Spike Encoding functions to extract the features we want.

In Statement \ref{st:window-width}, we demonstrated that each spike response has a distinct temporal extent. This 'window' signifies the period during which the spike can affect the downstream, or afferent, neurons it is connected to.

Let's use our definition of Rate Encoding from \ref{pscode:poisson-sample} to answer that question:

Given a Poisson process with rate $\lambda$, suppose we simulate the process as a series of $l$ coin tosses, where each coin toss corresponds to a time slot of duration $\Delta T$ in the measurement window. The chance of "success" (an event occurring) in each coin toss is $\lambda \Delta T$.

We can estimate the rate of the Poisson process as $\hat{R} = \frac{X}{T}$, where $X$ is the total number of "successes" (events) and $T = l \cdot \Delta T$ is the total measurement time.

