\subsection{Spike Encoders}

The arena of computational neuroscience has witnessed the advent of diverse methods for data encoding, especially in the transformation of conventional data formats into spatiotemporal patterns known as spike trains. These spike trains, sequences of action potentials, form the fundamental language of neurons and provide an effective medium for data encoding in neural models. The crucial step of converting classic data types into spike trains is pivotal to the functionality of these models, shaping how they perceive, process, and respond to input information.

In the forthcoming sections, we delve into some of the most prevalent techniques employed for encoding typical data types - including pixels from images or scalar values - into spike trains. Our exploration will encompass an array of encoding methods, each with its unique approach to transforming data into a neuronal language, catering to different kinds of applications and data characteristics. 

As we traverse through these encoding methodologies, the emphasis will be on understanding the underlying principles, the strengths and potential limitations of each technique, and their suitability under various contexts. The objective is to equip the reader with a robust comprehension of these encoding strategies, providing a springboard for their application in the development and analysis of neural models.

\subsection{Rate Encoding}

In rate encoding, the objective is to represent each feature pixel of an image as a sequence of discrete values, \(x_{i,j} \in \{0, 1\}\), while converting the image into a time-varying sequence of spikes that retain a relation to the original image. This process involves using a normalization function \(f: \mathbb{R} \rightarrow [0, 1]\) to map the input image \(x\) to a firing rate value \(p\) in the range \([0, 1]\) that is used to generate a stochastic spike sequence \(S_{i,j}\) using a chosen distribution \(D\).

One common implementation is to use the binomial distribution \(D = \text{Bin}(n, p)\), where \(n\) is the number of time samples, and \(p\) is an adjustable parameter according to the input value. The relationship between \(p\) and \(f(x_{i,j})\) is given by:

\begin{equation}
    p = f(x_{i,j}) \cdot F_0 \cdot \Delta t 
\end{equation}

where \(F_0\) is the maximum firing rate (20 Hz), and \(\Delta t\) is the time interval between consecutive samples. This equation allows us to set the expectancy \(E[D(x_{i,j})]\) for each \(x_{i,j}\) to \(f(x_{i,j}) \cdot F_0 \cdot T\), where \(T\) is the experiment time.

By using this derived equation, we can implement the rate encoding approach in a way that controls the spike generation based on the input image and ensures that high-intensity pixels (corresponding to higher values of \(f(x_{i,j})\)) will appear in more patterns.

Overall, the rate encoding approach, combined with the derived probability constraints and stochastic spike generation, enables us to effectively convert an image into a time-varying sequence of spikes while preserving its features. This method allows us to model the behavior of real neurons and capture the temporal dynamics of the input information, making it suitable for various applications in neural information processing and spiking neural network simulations.

\begin{figure}[H]
    \centering
    \includegraphics[width=0.5\textwidth]{methods/spike-encoding/graphs/rate-encoding-raster.png}
    \caption{Raster plot of a random Mnist image encoded via Rate Encoding with maximum rate of 20 Hz over trial time of 1 sec}
    \label{fig:rate-encoding-raster}
\end{figure}


\subsection{Latency Encoding}

\textbf{Latency Encoding:}

Let's have another look at the RC circuit model used to modulate the behavior of a LIF neuron. This model gives us an equation we have seen before:

\begin{figure}[H]
    \centering
    \includegraphics[width=0.25\textwidth]{methods/spike-encoding/graphs/LIF-circuit.png}
    \caption{RC Circuit}
    \label{image:RC_circuit}
\end{figure}

\begin{equation}
    I_{\text{in}} = I_R + I_C
\end{equation}

\begin{equation}
    I_{\text{in}} = \frac{1}{R} \cdot V_{\text{out}} + C \cdot \dot{V}_{\text{out}} \notag
\end{equation}

\begin{equation}
    \tau = R \cdot C \notag
\end{equation}


The result we computed earlier for constant input (for $v_{\text{th}} = 1, R = 1$) is:

\begin{equation}
    v(t) = I_{\text{in}} \cdot \left(1 - \exp\left(-\frac{t}{\tau}\right)\right)
\end{equation}

\begin{figure}[H]
    \centering
    \includegraphics[width=0.5\textwidth]{methods/spike-encoding/graphs/exponential-decay.png}
    \caption{time of discharge as a function of the input current}
    \label{image:exponential_decay}
\end{figure}

A spike occurs when $v(t) = v_{\text{threshold}} \implies t_{\text{spike}}(I_{\text{in}}) = -\tau \cdot \ln\left(\frac{I_{\text{in}}}{I_{\text{in}} - v_{\text{threshold}}}\right)$.

We can use that equation to encode $X \in \mathbb{R}_{\geq 0}^{M \times N}$ data into $T \in \mathbb{R}_{\geq 0}^{M \times N}$ latency vector, which represents the spike time of each input neuron. In other words, each input neuron represents a different pixel in the image.

Another thing we need to take into consideration is the normalization of the image's values before the encoding. We can do it by setting a normalization factor $I_0$ such that 
\begin{equation}
V_{\text{normalized}} = \frac{I_0}{\max(V)} \cdot V
\end{equation}

The trial's max time $T$ will serve as a threshold for small values because inputs whose values will inspire a spike after the trial ends will be ignored, i.e. $I_{\text{th}} = \frac{v_{\text{th}}}{1 - \exp\left(-\frac{T}{\tau}\right)} \implies \forall I > I_{\text{th}}, t_{\text{spike}}(I) > T$.

The minimum time $\Delta T$ which we allow a spike is:
\begin{equation}
I_0 = \frac{v_{\text{th}}}{1 - \exp\left(-\frac{\Delta T}{\tau}\right)}
\end{equation}

But a problem is at hand: the distribution of the values of each image is problematic.

\begin{figure}[H]
    \centering
    \includegraphics[width=0.8\textwidth]{methods/spike-encoding/graphs/mnist-values-histogram.png}
    \caption{histograms of the pixel values of random Mnist images divided into 10 bins}
    \label{fig:mnist-values-histogram}
\end{figure}

The fact that most values are 0's is not an issue; quite the opposite, we aim for sparsity. But the problem is that most values are distributed over 2-3 significant values. Another problem is that a linear normalization with logarithmic encoding is unable to distinguish close by values.

The resulting raster plot of the encoding expresses this issue:

\begin{figure}[H]
    \centering
    \includegraphics[width=0.8\linewidth]{methods/spike-encoding/graphs/latency-encoding-raster.png}
    \caption{raster plot of a random Mnist image encoded via latency encoding}
    \label{fig:latency-encoding-raster}
\end{figure}

To fix it, we suggest linear latency encoding by:
\begin{equation}
t(I_{\text{in}}) = -\tau \cdot \left(R \cdot I_{\text{in}} - V_{\text{thr}}\right)
\end{equation}

And clipping the zero values (they hold no useful information). The green lines represent the $\tau$ value, and the red lines represent $v_{\text{thr}}$.

\begin{figure}[H]
    \centering
    \includegraphics[width=0.5\linewidth]{methods/spike-encoding/graphs/exp-to-linear.png}
    \caption{on the right - the exponential latency encoding function, on the left - the liner latency encoding function}
    \label{fig:latency-exp-vs-lin}
\end{figure}

And as we can see, the results are significantly improved:

\begin{figure}[H]
    \centering
    \includegraphics[width=0.7\linewidth]{methods/spike-encoding/graphs/latency-encoding-raster-linear.png}
    \caption{ raster plot of a random Mnist image encoded via linear latency encoding}
    \label{fig:latency-encoding-raster-linear}
\end{figure}

\subsection{Population Encoding}

We shall now delve into various strategies to adopt the population encoding method. Our focus is to utilize this approach in a manner that leverages the salient features of the data, harnesses the unique characteristics of the neuron model, and employs the diverse encoding methodologies we have previously explored in \ref{ch:computational-models}.

As we discussed in section \ref{ch:computational-models}, we can integrate different Spike Encoding functions to extract the features we want.

In Statement \ref{st:window-width}, we demonstrated that each spike response has a distinct temporal extent. This 'window' signifies the period during which the spike can affect the downstream, or afferent, neurons it is connected to.

Let's use our definition of Rate Encoding from \ref{pscode:poisson-sample} to answer that question:

Given a Poisson process with rate $\lambda$, suppose we simulate the process as a series of $l$ coin tosses, where each coin toss corresponds to a time slot of duration $\Delta T$ in the measurement window. The chance of "success" (an event occurring) in each coin toss is $\lambda \Delta T$.

We can estimate the rate of the Poisson process as $\hat{R} = \frac{X}{T}$, where $X$ is the total number of "successes" (events) and $T = l \cdot \Delta T$ is the total measurement time.


