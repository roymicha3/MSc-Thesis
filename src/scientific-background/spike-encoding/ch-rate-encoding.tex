\subsection{Rate Encoding} \label{ssec:rate-encoding}

In rate encoding, the objective is to represent each feature pixel of an image as a sequence of discrete values, \(x_{i,j} \in \{0, 1\}\), while converting the image into a time-varying sequence of spikes that retain a relation to the original image. This process involves using a normalization function \(f: \mathbb{R} \rightarrow [0, 1]\) to map the input image \(x\) to a firing rate value \(p\) in the range \([0, 1]\) that is used to generate a stochastic spike sequence \(S_{i,j}\) using a chosen distribution \(D\).

One common implementation is to use the binomial distribution \(D = \text{Bin}(n, p)\), where \(n\) is the number of time samples, and \(p\) is an adjustable parameter according to the input value. The relationship between \(p\) and \(f(x_{i,j})\) is given by:

\begin{equation} \label{eq:rate-enc-prob}
    p = f(x_{i,j}) \cdot F_0 \cdot \Delta t 
\end{equation}

where \(F_0\) is the maximum firing rate (20 Hz), and \(\Delta t\) is the time interval between consecutive samples. This equation allows us to set the expectancy \(E[D(x_{i,j})]\) for each \(x_{i,j}\) to \(f(x_{i,j}) \cdot F_0 \cdot T\), where \(T\) is the experiment time.

Let's consider a \textbf{Poisson process} with rate parameter $R > 0$ in units of Hz (events per second). We aim to simulate this process over a time interval $T$ with a time-step $\Delta T$, both expressed in seconds.

The \textbf{Poisson process} is a type of counting process, where events occur independently and randomly in time. In a given small interval of length $\Delta T$, the probability of "success" (an event occurring) in this Poisson process is $R \Delta T$.

We divide the total time interval $T$ into $n = \frac{T}{\Delta T}$ smaller intervals. For each small interval, we generate a random number $U$ from the uniform distribution between 0 and 1. If $U$ is less than $R \Delta T$, we record a "success" (1, which means an event occurred). If not, we record a "failure" (0, which means no event occurred).

This procedure can be represented in pseudo code as follows:

\begin{mdframed}[backgroundcolor=green_background, linecolor=black, linewidth=2pt, frametitle=\textbf{Pseudo-code}] \label{pscode:poisson-sample}

function poisson process(R, T, $ \Delta T $): \\
  n = T / $ \Delta T $ \\
  events = zeroed array of size n \\
  for i in range(n):

\begin{enumerate}
    \item U = random number from uniform(0, 1)

    \item if $ U < R \cdot \Delta T \Rightarrow $ events[i] = 1
\end{enumerate}

return events

\end{mdframed}

This function returns a binary list of length $n$. Each entry of the list represents a small time interval of length $\Delta T$. If an event occurs in this time interval, the entry is 1, otherwise, it's 0.

This method assumes that $R \Delta T$ is small enough such that the probability of more than one event occurring within a small time interval $\Delta T$ is negligible.