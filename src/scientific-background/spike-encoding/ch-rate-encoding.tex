\subsection{Rate Encoding} \label{ssec:rate-encoding}

In rate encoding, the objective is to represent each feature pixel of an image as a sequence of discrete values, \(x_{i,j} \in \{0, 1\}\), while converting the image into a time-varying sequence of spikes that retain a relation to the original image. This process involves using a normalization function \(f: \mathbb{R} \rightarrow [0, 1]\) to map the input image \(x\) to a firing rate value \(p\) in the range \([0, 1]\) that is used to generate a stochastic spike sequence \(S_{i,j}\) using a chosen distribution \(D\).

One common implementation is to use the binomial distribution \(D = \text{Bin}(n, p)\), where \(n\) is the number of time samples, and \(p\) is an adjustable parameter according to the input value. The relationship between \(p\) and \(f(x_{i,j})\) is given by:

\begin{equation} \label{eq:rate-enc-prob}
    p = f(x_{i,j}) \cdot F_0 \cdot \Delta t 
\end{equation}

where \(F_0\) is the maximum firing rate (20 Hz), and \(\Delta t\) is the time interval between consecutive samples. This equation allows us to set the expectancy \(E[D(x_{i,j})]\) for each \(x_{i,j}\) to \(f(x_{i,j}) \cdot F_0 \cdot T\), where \(T\) is the experiment time.