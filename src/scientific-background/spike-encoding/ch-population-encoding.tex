\subsection{Population Encoding} \label{ssec{population-encoding}}

 \textit{population encoding} refers to a specific technique utilized in neural networks for encoding information about stimuli. The principle underpinning this technique is the collaborative function of a group of neurons, collectively encoding a single unit of information. This enables the capture of a greater complexity of information than what could be represented by individual neurons.

A key advantage of population encoding is the ability to represent a more diverse array of stimulus features. Under this scheme, each neuron in the network can be assigned distinct parameters, which are determined by the specific model and encoding method used. As a result, each neuron in the population can respond differently to the same stimulus, thereby facilitating the encoding of various aspects of the stimulus.

Population encoding methods can incorporate both \textit{Rate Encoding} and \textit{Latency Encoding}, thus combining the benefits of both methods and more. However, it is worth noting that the choice between Rate and Latency encoding, or a combination of both, should be chosen by the specifics of the problem at hand.

To formally define population encoding, let us consider $M$ to be the number of neurons in the population and $\{ E_m \}_{m=1}^{M}$ to be the set of encoders within the population, each corresponding to a specific neuron.

Given a non-negative real input $x \in \mathbb{R}_{\geq 0}$ and a trial time of $T$ time samples, the population encoding method is given by:

\begin{equation}
    PE(x) = 
    \begin{bmatrix} E_{1}(x) \\ E_{2}(x) \\ \vdots \\ E_{M}(x) \end{bmatrix} \in \mathbb{R}^{M \times T}
\end{equation}

This equation provides a general form of population encoding, where each encoder $E_m(x)$ applies to the input $x$ and produces a time series of outputs, resulting in a $M \times T$ matrix representation.

From the perspective of an efferent neuron, the duration for assessing the afferent spike rate is crucial for accurate interpretation. Assuming the spike generation obeys a Poisson distribution, the confidence interval for the actual coded frequency, $F$, can be evaluated based on the observed frequency, $f = \frac{\text{spikes}}{\Delta T}$. A limited evaluation time, $\Delta T$, potentially introduces inaccuracies in this process. The degree of such inaccuracies directly influences the fidelity of information transmission across the network.

The significance of neural activity is not just about whether spikes occur but also how frequently they occur. Accordingly, it's crucial for our models to interpret these spike rates in order to make informed decisions.

When the assessment duration is short, the observed spike frequency may not accurately represent the true spike rate due to the inherent stochastic nature of spike generation, leading to misinterpretations. Consequently, the transmitted information can be distorted or lost, affecting the overall network's performance.

In our analysis, we followed the argument put forth by \cite{gautrais1998rate}, claiming that if a single spike is observed within a 10 ms timeframe, the most that can be inferred, with 90\% confidence, is that the actual spike frequency lies between 5 and 474 Hz.

We now turn our attention to how we can incorporate this information within a given time range. It's crucial to note that observing one Poisson process during a span of $\Delta T$ ms is the same as observing n such processes during a period of $\frac{\Delta T}{n}$ ms.

If we consider an afferent firing rate of 20 Hz, to make a valid assumption, we need to at least sample over a window of 50 ms, or alternatively, a window of 50/n ms spread across n neurons. However, the question remains: is this enough?