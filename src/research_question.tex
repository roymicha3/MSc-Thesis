\chapter{Research Question}
\label{chap:rq}

\section*{Research Question 1}
\textit{How can gating mechanisms be integrated into the Tempotron model to enhance the training efficiency and performance of Spiking Neural Networks?}

\textbf{Elaboration:}
This research question aims to investigate the integration of gating mechanisms, Gating Units, and different cost functions into the Tempotron model for training Spiking Neural Networks (SNNs). The study will explore how the inclusion of gating mechanisms can enhance the network's ability to capture temporal patterns and context in spiking patterns. The research will involve the design and implementation of novel variations of the Tempotron model that incorporate gating mechanisms. Training efficiency and performance metrics, such as classification accuracy, sparsity, and computational resources, will be compared between the original Tempotron model and the modified versions with gating. The investigation will also focus on understanding the impact of gating mechanisms on training stability and generalization in SNNs.

\section*{Research Question 2}
\textit{What are the most effective backpropagation techniques that can be adapted and applied to the Tempotron model for training Spiking Neural Networks, and how do they compare in terms of convergence speed and accuracy?}

\textbf{Elaboration:}
This research question aims to explore various backpropagation techniques adapted for training SNNs using the Tempotron model. The study will evaluate the convergence speed and accuracy achieved by each backpropagation variant during the training process. Comparisons will be made based on their performance across different datasets and SNN architectures. Additionally, the research will delve into the computational requirements and memory usage of each backpropagation technique to assess their suitability for real-time and energy-efficient applications in SNNs.

\section*{Research Question 3}
\textit{What strategies can be employed to overcome the computational challenges of training large-scale Spiking Neural Networks in the context of the Tempotron model, and how do these strategies impact the network's performance and resource requirements?}

\textbf{Elaboration:}
The performance and resource requirements of SNNs trained with these strategies will be compared with conventionally trained networks. The research will also explore how these approaches impact the network's accuracy and sparsity. Additionally, considerations will be given to the scalability of these strategies for even larger SNN architectures, paving the way for more practical and resource-efficient implementations.


